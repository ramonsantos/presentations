\begin{frame}[fragile]{Load balancing}
    \begin{itemize}%[<+->]
        Load balancing (balanceamento de carga) é o processo de redistribuição da carga de trabalho entre os nós de um sistema distribuído para melhorar tanto a utilização dos recursos, quanto o tempo de resposta das tarefas, evitando também situações em que alguns nós fiquem muito carregados enquanto outros fiquem ociosos ou realizando pouco trabalho.
    \end{itemize}
\end{frame}


\begin{frame}[fragile]{Características de um load balancing}
    \begin{itemize}%[<+->]
        \item Distribuir tráfego/carga entre múltiplos servidores.
        \item Possibilita a escalabilidade horizontal.
        \item Segurança: Separa tráfego público de tráfego privado.
        \item Pode atuar nas camadas de aplicação e transporte.
        \item Realizam health checks para verificar se um servidor consegue receber requisições.
        \item Possibilita manutenção de servidores sem downtime.
    \end{itemize}
\end{frame}


\begin{frame}[fragile]{Mecanismos de balanceamento}
    \begin{itemize}%[<+->]
        \item \textbf{Round Robin:} Envia requisições sequencialmente entre os servidores.
        \item \textbf{Least Connections:} Escolhe o servidor com menos conexões ativas.
        \item \textbf{IP Hash:} Uma função hash é usada para determinar qual servidor deve receber a próxima request com base no endereço IP do cliente.
        \item \textbf{Weighted load balancing:} Usa o algoritmo Round Robin com diferentes pesos nos servidores. É útil quando determinado servidor conta com mais capacidade computacional que os demais.
        \item \textbf{Least Time:} Escolhe o servidor com o menor tempo médio de resposta.
    \end{itemize}
\end{frame}

\begin{frame}[fragile]{Ferramentas}
    \begin{itemize}%[<+->]
        \item \textbf{Nginx:} Leve e extensível (com Lua). Muito usado para balancear carga HTTP.
        \item \textbf{HAProxy:} Além de ser usado para balanceamento de carga na camada de aplicação, é bastante popular no balanceamento de carga na camada de transporte. Como, por exemplo, balanceamento de conexão de leitura entre réplicas read-only do PostgreSQL.
        \item \textbf{Cloudflare Load Balancing:} É um serviço de balanceamento de carga global, que tem como principal característica o um roteamento inteligente e personalizável, onde a região geográfica, ou até mesmo as coordenadas do GPS do dispositivo que fez a requisição, podem ser considerados para encontrar o nó de menor latência.
        \item \textbf{AWS Elastic Load Balancing:}
        \begin{itemize}
            \item \textbf{Application Load Balancer:} Balanceador de carga da camada de aplicação.
            \item \textbf{Network Load Balancer:} Balanceador de carga da camada de transporte.
        \end{itemize}
    \end{itemize}
\end{frame}