\begin{frame}[fragile]{API Gateway}
    \begin{itemize}%[<+->]
        Um API Gateway é um ponto de entrada único para gerenciar e encaminhar requisições a diferentes serviços de uma aplicação em um sistema distribuído.
    \end{itemize}
\end{frame}


\begin{frame}[fragile]{Características de um API gateway}
    \begin{itemize}%[<+->]
        \item Pode fazer balanceamento de carga.
        \item Atua na camada de aplicação.
        \item Faz tradução de protocolo.
        \item Monitoramento, Logging e Analytics.
        \item Gerenciamento da API:
            \begin{itemize}
                \item roteia requisições para serviços mapeados;
                \item versionamento de API;
                \item adiciona headers e query params;
                \item throttling e rate limiting;
                \item transformação e agregação de resposta.
            \end{itemize}
        \item Diminui carga de trabalho nos serviços:
            \begin{itemize}
                \item pode responder a requisições com dados em cache;
                \item pode lidar com autenticação e autorização;
                \item validação de requisições;
                \item não repassaria requisições de um ataque DDoS para serviços.
            \end{itemize}
    \end{itemize}
\end{frame}


\begin{frame}[fragile]{Desafios e desvantagens}
\begin{itemize}%[<+->]
        \item Adiciona uma nova camada da arquitetura:
            \begin{itemize}
                \item mais um componente para o time manter;
                \item mais ferramentas que as pessoas do time devem conhecer;
                \item cada request passa por uma nova camada, o que pode gerar latência adicional;
                \item aumenta o custo financeiro com ferramentas gerenciadas ou infra em soluções self-hosted.
            \end{itemize}
        \item É um ponto único de falha:
            \begin{itemize}
                \item se o gateway cair, todos os serviços ficam inoperantes;
                \item erros de configuração podem comprometer segurança e desempenho;
                \item é necessário buscar alta disponibilidade, mediante técnicas como replicação e balanceamento, ou usar serviços gerenciados que suportem esse requisito.
            \end{itemize}
    \end{itemize}
\end{frame}

\begin{frame}[fragile]{Ferramentas}
    \begin{itemize}%[<+->]
        \item \textbf{Kong:} Funciona com Nginx e é extensível através de plugins escritos em Lua.
        \item \textbf{AWS API Gateway:} Serviço de API Gateway gerenciado da AWS.
    \end{itemize}
\end{frame}